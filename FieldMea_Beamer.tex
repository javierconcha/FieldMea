\documentclass{beamer}
\usepackage[latin1]{inputenc}
\usepackage{textpos}

\usepackage{graphics}
% \usepackage[demo]{graphicx}
\usepackage{adjustbox}

\usepackage[english]{babel}
\usepackage{colortbl}
\usepackage{caption}
% \usepackage{subcaption}
\usepackage{multirow}
\usepackage{amsmath}
\usepackage[makeroom]{cancel}
\usepackage{xcolor} % for colored text

\usepackage{tikz} % for flow charts
\usetikzlibrary{shapes,arrows,positioning,shadows,calc}

\usepackage{filecontents}% http://ctan.org/pkg/filecontents
\usepackage{silence}% http://ctan.org/pkg/silence
\WarningFilter{latex}{Overwriting file}% Remove LaTeX warnings starting with "Overwriting file"
\begin{filecontents*}{linereg.data}
#x y
0 4
10 24
\end{filecontents*} 

\begin{filecontents*}{linereg2.data}
#x y
2 8
8 20
\end{filecontents*} 

	\renewcommand<>{\item}[1]{\only#2{\beameroriginal{\item}{#1}}} % for replace a equation for other equation in the same place
	
	% \usetheme{Warsaw}
	\usetheme{Frankfurt}
	% \usetheme{Boadilla}
	\setbeamertemplate{navigation symbols}{} 
	% \useoutertheme{infolines} 
% \setbeamertemplate{footline}{\hbox{\vspace{0.1cm} \insertshortauthor \hspace*{3.5cm} \insertshorttitle \hspace*{4.0cm} \hfill\insertframenumber/\inserttotalframenumber}} 

\setbeamertemplate{footline}{\hbox{\vspace{0.1cm} \insertshortauthor \hspace*{3.5cm} \insertshorttitle \hspace*{4.4cm} \hfill\insertframenumber}} 

\def\braces#1{[#1]} % to define square parenthesis 
	
% \usecolortheme{orchid}

% \usecolortheme{lily}

% \usecolortheme{default}
\usecolortheme{cranejavier}

% \setbeamertemplate{footline}[frame number]
% \setbeamertemplate{footline}[page number]
	
	
	% -------------------------------------- Slide 1
	\title[Field Measurements and Instrumentation]{Intro to Instrumentation and Field Measurements in Remote Sensing}
	\author[J. Concha \& P. Romanczyk]{\Large Javier Concha and Paul Romanczyk}
	\institute{\footnotesize Digital Imaging and Remote Sensing Lab\\Chester F. Carlson Center for Imaging Science\\ Rochester Institute of Technology}
	\date{\today}


\AtBeginSection[ ]
{	\setbeamertemplate{footline}{} 	
	\begin{frame}{\LARGE Outline} 
	\LARGE
		\tableofcontents[currentsection]
	\end{frame}

\addtocounter{framenumber}{-1}	

\setbeamertemplate{footline}{\hbox{\vspace{0.1cm} \insertshortauthor \hspace*{2.0cm} \insertshorttitle \hspace*{3.8cm} \hfill\insertframenumber}} 
}	

% \AtBeginSubsection[ ]
% {		
% 	\begin{frame}{\LARGE Outline} 
% 		\tableofcontents[currentsection,currentsubsection]
% 	\end{frame}
% \addtocounter{framenumber}{-1}	
% }		
\newcounter{tmpc} % for resume counter
%&&&&&&&&&&&&&&&&&&&&&&&&&&&&&&&&&&&&&&&&&&&&&&&&&&&&&&&&&&&&&&&
%&&&&&&&&&&&&&&&&&&&&&&&&&&&&&&&&&&&&&&&&&&&&&&&&&&&&&&&&&&&&&&&
\begin{document}
{	
\setbeamertemplate{footline}{} 
\setbeamertemplate{headline}{}
	
	\begin{frame} 
	\titlepage
	
	\begin{textblock*}{10cm}(10.0cm,-8.2cm)
	   \includegraphics[height=10mm]{Images/tiger_walking_rit_color.eps}
	\end{textblock*}
	
	\begin{textblock*}{10cm}(-.7cm,-8.2cm)
	   \includegraphics[height=10mm]{./Images/dirs_logo.png}
	\end{textblock*}
	
	\begin{textblock*}{9cm}(2cm,-4.5cm)

	   \tikz\node[opacity=0.3]{ \includegraphics[width=65mm]{./Images/landsat8-earth.jpg}};
	\end{textblock*}

	\begin{textblock*}{12cm}(3.0cm,0cm)
	   \scriptsize Presented for 2015 Intersession Term
	\end{textblock*}
	
	\end{frame}

}
\addtocounter{framenumber}{-1}
%\setbeamercovered{highly dynamic}
%\setbeamercovered{transparent}
\setbeamercovered{still covered={\opaqueness<1->{2}},again covered={\opaqueness<1->{2}}}

% ----------------------------------- Slide ----------------------------------------------	

\addtobeamertemplate{frametitle}{}{%
\begin{textblock*}{90mm}(8.2cm,-0.5cm)
% \includegraphics[height=0.5cm]{/Users/javier/Desktop/Javier/MASTER_RIT/SPIE2012/Slides/rit_white_no_bar.jpg}
\includegraphics[height=0.4cm]{./Images/RIT_LOGO.png}
\end{textblock*}}


% ----------------------------------- Slide ----------------------------------------------
{
\setbeamertemplate{footline}{} 
\begin{frame}{\LARGE Outline} 
\LARGE
	\tableofcontents
\end{frame}

\addtocounter{framenumber}{-1}
}

%%%%%%%%%%%%%%%%%%% SECTION %%%%%%%%%%%%%%%%%%%%%%%%%%%%%%%%
\section{Introduction}
\subsection*{Motivation}
% --- slide ------------------------------------------------
\begin{frame}{\LARGE Course Goals} 
\LARGE
\begin{itemize}\itemsep.4cm
	\item Learn the importance of field measurements
	\item Learn how to take field measurements
	\item Learn about DIRS instruments
\end{itemize}
\end{frame}
% --- slide ------------------------------------------------
\begin{frame}{\LARGE Course Description} 
\LARGE
\begin{itemize}\itemsep.4cm
	\item Friday: Introduction
	\item Monday: Introduction (con't) and DIRS instruments exhibition
	\item Tuesday: Lab: Reflectance measurements
	\item Wednesday: Lab: LIDAR measurements
\end{itemize}
\end{frame}
% --- slide ------------------------------------------------
\begin{frame}{\LARGE Definitions} 
\Large

\begin{itemize}\itemsep.5cm
	\item {\bf Remote Sensing:}\\
``Remote sensing is the science of obtaining information about objects or areas from a distance, typically from aircraft or satellites.''\\
	\item {\bf Field Measurements or Groundtruth or Ground-based data or reference data or ancillary data:}\\

``Observations or measurements made at or near the surface of the earth in support of remote sensing.''\\

\end{itemize}

\end{frame}
% --- slide ------------------------------------------------
\begin{frame}{\LARGE Motivation}
\LARGE
{\bf Why is it important?}\\
\begin{itemize}\itemsep.3cm
	\item Validation: comparison to know how close a model is to the field measurements (accuracy)
	\item Calibration or Correction: adjust model or instrument to be more precise (data fitting)
	\item Data collection to get characteristics of target, materials, etc.
\end{itemize}
\end{frame}
% --- slide ------------------------------------------------
\begin{frame}{\LARGE Motivation}{\vspace{0.1cm} \Large Examples}
\vspace{-.7cm}
Include:\\
Javier's example (over water mea.)\\
Paul's example (LIDAR and trees?)

\end{frame}
% --- slide ------------------------------------------------
\begin{frame}{\LARGE Field Data Collection}{\vspace{0.1cm} \Large Area of Study}
\begin{figure}[htb]
  \centering
	\includegraphics[height=7cm]{./Images/AreaOfStudy1.pdf} 
  % \caption{Sites in the Rochester Embayment for the water sample collection on September, $19^{th}$, 2013.\label{fig:0910913Sites} } 
\end{figure}
\end{frame}
% --- slide ------------------------------------------------
\begin{frame}{\LARGE Field Data Collection (con't)}{\vspace{0.1cm} \Large Area of Study}
\begin{figure}[htb]
  	\centering
  	\includegraphics[height=6cm]{./Images/AreaOfStudy2.pdf}
\end{figure}
\end{frame}
% --- slide ------------------------------------------------
\begin{frame}{\LARGE Field Data Collection (con't)}
\vspace{-1cm}
\begin{figure}[htb]
\centering
\includegraphics[height=7cm]{./Images/Collection.pdf}
      
\end{figure}
% \centerline{Comparison between traditional ELM (dashed lines)}
% \centerline{and model-based ELM (solid lines).}
\end{frame}
% --- slide ------------------------------------------------
\begin{frame}{\LARGE Field Data Collection (con't)}{\Large Lab Measurements} 
\vspace{-1cm}
\begin{figure}[htb]
\centering
\includegraphics[height=7cm]{./Images/LabMeasurements.pdf}
      
\end{figure}
% \centerline{Comparison between traditional ELM (dashed lines)}
% \centerline{and model-based ELM (solid lines).}
\end{frame}
% --- slide ------------------------------------------------
\begin{frame}{\LARGE Field Data Collection (con't)}{\vspace{0.1cm} \Large 2013 and 2014 Seasons}
\begin{table}[htb]
  
  \centering
  \includegraphics[width=11cm]{./Images/Collect1314.png}
  \label{tab:collect}
\end{table}
\end{frame}
%%%%%%%%%%%%%%%%%%% SECTION %%%%%%%%%%%%%%%%%%%%%%%%%%%%%%%%
\section{Background}
\subsection*{}
% --- slide ------------------------------------------------
\begin{frame}{\LARGE Examples of Kind of Measurements} 
\Large
\begin{itemize}\itemsep.4cm
	\item Reflectance: Radiometer
	\item Concentration: Spectrophotometer
	\item Location: GPS
	\item Structure: LIDAR
	\item Leaf Area Index (LAI): Ceptometer
\end{itemize}

\end{frame}
% --- slide ------------------------------------------------
\begin{frame}{\LARGE Radiometric Quantities: Radiance} 
% \begin{frame}{Radiometric Quantities: Radiance}
\begin{figure}[H]
\begin{columns}[t] % contents are top vertically aligned
	\begin{column}[T]{6cm} % each column can also be its own environment
  		\includegraphics[height=6cm]{./Images/RadianceDef.png}
    \end{column}
	\begin{column}[T]{4cm} % each column can also be its own environment
  		$\Delta Q$: radian energy incident \\
  		$\Delta t$: time interval \\
  		$\Delta A$: surface area at location (x,y,z)\\
  		$\Delta\Omega$: solid angle in direction ($\theta$,$\varphi$) \\
  		$\Delta\lambda$: photons wavelength interval
    \end{column}
\end{columns}	
\end{figure}
\only<1>{\begin{equation}
	\small L(x,y,z,t,\theta,\varphi,\lambda)\equiv\frac{\Delta Q}{\Delta t\Delta A\Delta\Omega\Delta\lambda}~~\left[ Js^{-1}m^{-2}sr^{-1}nm^{-1} \right]
		\end{equation}}
\only<2>{\begin{equation}
	\small L(x,y,z,t,\theta,\varphi,\lambda)\equiv\frac{\partial^4 Q}{\partial t\partial A\partial\Omega\partial\lambda}~~\left[ Js^{-1}m^{-2}sr^{-1}nm^{-1} \right]
		\end{equation}}
\end{frame}
% --- slide ------------------------------------------------
\begin{frame}{\LARGE Radiance Sensor} 
\begin{figure}
\centering
    \includegraphics[height=5cm]{./Images/RadianceSensor.png}
\end{figure}
\vspace{-.7cm}
\hfill \scriptsize (Source: \cite{MobleyOnline})

% \vspace{0.2cm}
% \centerline{Water Pixels}
% \centerline{(Unknown concentrations)}

\end{frame}
% --- slide ------------------------------------------------
\begin{frame}{Radiometric Quantities: Irradiance}
\textbf{Spectral downwelling scalar irradiance} at depth z:
\begin{equation}
	E_{od}(z,\lambda)=\int_{2\pi_d} L(z,\theta,\varphi,\lambda)d\Omega~~\left[Wm^{-2}nm^{-1} \right]
\end{equation}
\textbf{Spectral upwelling scalar irradiance} at depth z:
\begin{equation}
	E_{ou}(z,\lambda)=\int_{2\pi_u} L(z,\theta,\varphi,\lambda)d\Omega~~\left[Wm^{-2}nm^{-1} \right]
\end{equation}
\textbf{Spectral scalar irradiance} at depth z:
\begin{align}
	E_{o}(z,\lambda) &\equiv E_{od}(z,\lambda)+E_{ou}(z,\lambda)\\
					 &=\int_{4\pi} L(z,\theta,\varphi,\lambda)d\Omega
\end{align}
\end{frame}
% --- slide ------------------------------------------------
\begin{frame}{\LARGE Scalar Irradiance Sensor} 
\begin{figure}
\centering
    \includegraphics[height=5cm]{./Images/ScalarIrradianceSensor.jpg}
\end{figure}
\vspace{-.7cm}
\hfill \scriptsize (Source: \cite{MobleyOnline})

% \vspace{0.2cm}
% \centerline{Water Pixels}
% \centerline{(Unknown concentrations)}

\end{frame}
% --- slide ------------------------------------------------
\begin{frame}{Radiometric Quantities: Irradiance}
\textbf{Spectral downwelling plane irradiance} at depth z:
\begin{equation}
	E_{d}(z,\lambda)=\int_{2\pi_d} L(z,\theta,\varphi,\lambda)|cos\theta|d\Omega~~\left[Wm^{-2}nm^{-1} \right]
\end{equation}
Photosynthetic available radiation, \textbf{PAR}:
\begin{equation}
	PAR(z)\equiv \int_{350nm}^{700nm} \frac{\lambda}{hc}E_o(z,\lambda)d\lambda~~~\left[photons~s^{-1}m^{-2} \right]
\end{equation}
\end{frame}
% --- slide ------------------------------------------------
\begin{frame}{\LARGE Planar Irradiance Sensor} 
\begin{figure}
\centering
    \includegraphics[height=5cm]{./Images/PlaneIrradianceSensor.png}
\end{figure}
\vspace{-.7cm}
\hfill \scriptsize (Source: \cite{MobleyOnline})

% \vspace{0.2cm}
% \centerline{Water Pixels}
% \centerline{(Unknown concentrations)}

\end{frame}
% --- slide ------------------------------------------------
\begin{frame}{Reflectance}
\begin{itemize}\itemsep.4cm
	\item {\textbf{Irradiance reflectance:}\\
			\begin{equation}
				R(z,\lambda)\equiv \frac{E_u(z,\lambda)}{E_d(z,\lambda)}
			\end{equation}}\\
	\item{ \textbf{Remote sensing reflectance (water):}\\
			\begin{equation}
				R_{rs}(\theta,\varphi,\lambda)\equiv \frac{L_w(\theta,\varphi,\lambda)}{E_d(\lambda)}~~\left[sr^{-1} \right]
			\end{equation}\\
			where $L_w$ is the \textbf{water-leaving radiance}\\}
	\item{ \textbf{Bidirectional Reflectance Distribution Function (BRDF):}\\
			\begin{equation}
				r_{BRDF} = \frac{\displaystyle L(\theta_o,\phi_o)}{\displaystyle E(\theta_i,\phi_i)}~~~\left[sr^{-1}\right]
			\end{equation}}
\end{itemize}
\end{frame}
%%%%%%%%%%%%%%%%%%% SECTION %%%%%%%%%%%%%%%%%%%%%%%%%%%%%%%%
\section{Taking Measurements}
\subsection*{}
% --- slide ------------------------------------------------
\begin{frame}{Diffuse white reference panel (Spectralon)}
\begin{columns}[c] % contents are top vertically aligned
  	\begin{column}[T]{5cm}

		\begin{itemize}\itemsep.4cm
			\item{For a Lambertian surface:\\
				\vspace{.1cm}
				$L = \frac{\displaystyle E_dr}{\displaystyle \pi} \Rightarrow E_d = \frac{\displaystyle  L\pi}{\displaystyle r}$}
			\item{ For spectralon $r\approx 1$ ($\approx 100\%$)\\
			$\Rightarrow E_d = L\pi$}
		\end{itemize}
	\end{column}	
	\begin{column}[T]{7cm} % each column can also be its own environment
	  			\begin{figure}[htb]
					\centering
					 \includegraphics[height=3cm]{./Images/spectralon-web.jpg}
				\end{figure}
				\begin{figure}[htb]
					\centering
					 \includegraphics[height=3cm]{./Images/SpectralonTargetsReflectance.jpg}
				\end{figure}
	 
	\end{column}
\end{columns}
\end{frame}
% --- slide ------------------------------------------------
\begin{frame}{Reflectance}
\begin{itemize}\itemsep0.4cm
	\item{ \textbf{Reflectance:} $R(z,\lambda)\equiv \frac{\displaystyle L_u(z,\lambda)}{\displaystyle E_d(z,\lambda)}$}
	\item Two measurements:\\
		\begin{itemize}\itemsep.2cm
			\item{ $E_d$: Spectralon}
			\item{ $L_u$: Target or sample}
		\end{itemize}
\end{itemize}



\end{frame}
% --- slide ------------------------------------------------
\begin{frame}{\LARGE Remote-Sensing Reflectance}
 	\begin{columns}[c] % contents are top vertically aligned
  		\begin{column}[T]{5cm} % each column can also be its own environment
  		\begin{itemize}
  		\item 3 measurements:
	  		\begin{itemize}
	  			\item {$L_g$ (spectralon)}
	  			\item {$L_t = L_r+L_w$ (water)}
	  			\item {$L_{sky}$}
	  		\end{itemize}
	  	\item{Remote-sensing reflectance:\\
	  			\vspace{-0.5cm}
	  			\begin{gather*} 
	  				R_{rs} = L_w/E_d\\
	  				= (L_t - L_r)/E_d
	  			\end{gather*}\\
	  			with $L_r = 0.028*L_{sky}$}	
	  	
	  	\item {$E_d = L_g*\pi$}
	  	\item {$\phi:$ azimuthal angle}
	  	\item {$\theta:$ zenith angle}
  		\end{itemize}
    	\end{column}
    	\hspace{-1cm}
    	\begin{column}[T]{7cm} % each column can also be its own environment
  			\begin{figure}[htb]
				\centering
				 \includegraphics[height=6cm]{./Images/LwGeometry.png}
			\end{figure}
 
    	\end{column}
    \end{columns}
\end{frame}

% --- slide ------------------------------------------------
\begin{frame}{$L_{sky}$}

\begin{figure}[htb]
				\centering
				 \includegraphics[height=6cm]{./Images/LskyGeometry.png}
			\end{figure}

\end{frame}
% --- slide ------------------------------------------------
\begin{frame}{Things to consider for a field campaign}

\begin{itemize}

	\item have a good datasheet
	\item download and look data ASAP
	\item tools for in-field repairs and cleaning
	\item water, food, field clothes, first aid
	\item plan collection ahead of time
	\item write down procedure
	\item setting up plots/sites
	\item take pictures
	\item read manual
	\item charge batteries
	\item check batteries
	\item check instrument in advance
	\item The data should be taken as close to the time of acquisition as possible (due to inevitable landscape changes)

\end{itemize}

\end{frame}
% --- slide ------------------------------------------------
\begin{frame}{Field Campaigns}

\begin{itemize}
	\item Brings theory to practice
	\item It is FUN!
\end{itemize}

SHOW PICTURES HERE!

\end{frame}
%%%%%%%%%%%%%%%%%%% SECTION %%%%%%%%%%%%%%%%%%%%%%%%%%%%%%%%
\section{DIRS Instruments}
\subsection*{}
% --- slide ------------------------------------------------
\begin{frame}{DIRS Lab Instruments}
\begin{itemize}
	\item ASD
	\item SVC
	\item RIT LIDAR
	\item GRIT
	\item WASP
	\item GPS
	\item HydroScat-2
	\item AccuPAR LP-80
	\item WASP
	\item WASP Lite
	\item MISI
	\item Thermistor 
	\item Camera
	\item Tape measures
\end{itemize}

\end{frame}
% --- slide ------------------------------------------------
\begin{frame}{Data Extraction}
\begin{itemize}\itemsep.4cm
	\item Raw data in proprietary format (depending of the instrument). Ex: *.dat, *..SPC
	\item Translate or export to ASCII (*.txt, *.ASC)
	\item Text files manipulation (IDL, Matlab, Python, C++, bash).
\end{itemize}



\end{frame}
% --- slide ------------------------------------------------
\begin{frame}{Data in ASCII format}{ASD}

\begin{figure}
\centering
    \includegraphics[height=7cm]{./Images/TextFileASD.png}
\end{figure}

\end{frame}
% --- slide ------------------------------------------------
\begin{frame}{Data in ASCII format}{SVC}
\vspace{-.5cm}
\begin{figure}
\centering
    \includegraphics[height=7cm]{./Images/TextFileSVC.png}
\end{figure}

\end{frame}
% --- slide ------------------------------------------------
\begin{frame}{Reflectance from Matlab}{SVC}
\begin{figure}
\centering
    \includegraphics[height=6cm]{./Images/Reflectances.png}
\end{figure}

\end{frame}
%%%%%%%%%%%%%%%%%%% SECTION %%%%%%%%%%%%%%%%%%%%%%%%%%%%%%%%
% \section{Conclusions}
% \subsection*{Conclusions}
% % --- slide ------------------------------------------------
% \begin{frame}{\LARGE Conclusions}

% \Large
% \begin{itemize}\itemsep.4cm
% 	\item Current retrieval algorithm depends on IOPs from the field. Not always available!

% 	\item LUT from Hydrolight: Highly dependent in phase function

% 	\item Obtain field data for Landsat-8 is difficult, mainly for weather conditions
% \end{itemize}

% \end{frame}

% --- slide ------------------------------------------------
{	
\setbeamertemplate{footline}{} 
\setbeamertemplate{headline}{}
\begin{frame}[noframenumbering] 

\vspace{\baselineskip}
\centerline{\Large Thanks for your attention!}
	\vspace{\baselineskip}
\vspace{-.3cm}
% \centerline{\Huge QUESTIONS?}
\uncover <2->{\centerline{\Huge QUESTIONS?}}
\vspace{\baselineskip}
\centerline{Javier A. Concha}
\centerline{jxc4005@rit.edu}

\begin{figure}[htb]
\centering
\includegraphics[height=5cm]{./Images/LC80160302013262LGN00.jpg}
\end{figure}
\vspace{-.5cm}
\centerline{\tiny (09/19/2013)}
\end{frame}
}
% % BIBLIOGRAPHY
% \bibliographystyle{apalike}

% \bibliography{/Users/javier/Desktop/Javier/PHD_RIT/Latex/javier_bib}
% --- slide ------------------------------------------------
\section*{}
\begin{frame}%[shrink=30] 
\tiny
  \frametitle{References}
  % \nocite{*}
  \bibliographystyle{apalike}
  \bibliography{./javier_bib}
\end{frame}
\end{document} 
% EEEEEEEEEEENNNNNNNNNNNNNNDDDDDDDDDDDD